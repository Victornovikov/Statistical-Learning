глава 2
упражнения

1 а) негибкая лучше
б) негибкая лучше ?
в) гибкая лучше
г)негибкая лучше

2

# глава 3 -- вопросы

Связь дисперсии и SE.  
(страница 65 учебника, формула 3.7)

мы ведь(в общем случае) работаем с
выборочным средним / оценкой дисперсии? Не понимаю смысл этой
величины. 
Когда считаем σ  мы ведь уже делим на n-1. Почему в SE мы еще раз делим на n? 

На стр66 сказано "standard error tells us the average amount that this estimate ˆ μ differs from the actual value of μ."
Когда тут говорится "среднее отклонение", оно считается по какой "выборке"? мы выбираем ного "подвыборок" и смотрим, как варьируется среднее?

---

стр 66 сразу после формулы 3.8
For these formulas to be strictly valid, we need to as-sume that the errors � i for each observation are uncorrelated with common variance σ2. This is clearly not true in Figure 3.1, but the formula still turns out to be a good approximation.

Не понимаю, о чем конкретно они и как это "очевидно" из рисунка.

--

Откуда берется деление на n-2 в формуле RSE? Предполагаю, что это связано со степенями свободы (которые я толком так никогда и не понял) -- этот вопрос наверное можно опустить, не уверен что его стоит сейчас обсуждать

--

Мы можем каким-то образом посчитать confidence interval для бета. Как мы зная интервалы для бета считаем интервал для оценки Y?

--

Тесты. t-тест, F-тест.
я (вроде) понимаю идею того как работают тесты, но было бы хорошо обсудить еще раз, что происходит. 
Как мы решаем отвергается гипотеза или нет. Как мы формулируем гипотезу. Как выбираем тест. 
Как связаны т-тест и ф-тест.
Это большая тема на нее можно отедльное занятие, сейчас наверное было бы просто хорошо проговорить, что происходит в контексте регрессии.